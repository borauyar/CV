\documentclass[11pt,a4paper,sans]{moderncv}

% ModernCV themes
\moderncvstyle{classic} % Style options: 'casual', 'classic', 'banking', 'oldstyle', 'fancy'
\moderncvcolor{blue}    % Color options: 'blue', 'orange', 'green', 'red', 'purple', 'grey', 'black'

% Encoding and fonts
\usepackage[utf8]{inputenc}
\usepackage[T1]{fontenc}

% Adjust margins
\usepackage[scale=0.75]{geometry}

% Personal data
\name{Dr. Bora}{Uyar}
\title{Senior Bioinformatics Scientist @ MDC-Berlin}                   % Optional
\address{Hannoversche Str. 28}{Berlin 10115}{Germany}% Optional
\email{bora.uyar@mdc-berlin.de}                    % Optional
\homepage{https://borauyar.github.io/}                        % Optional
\social[github]{borauyar}                          % Optional
\social[linkedin]{bora-uyar-11050425}              % Optional
\photo[96pt][0.8pt]{BU.jpg}                        % Optional, adjust dimensions as needed
\quote{}                                           % Optional

\begin{document}
\makecvtitle

\section{Summary}
I am a Senior Staff Scientist at the Bioinformatics \& Omics Data Science platform of the Max-Delbrück-Center for Molecular Medicine. My current research is focused on the development of deep learning-based multi-modal data integration tools for precision oncology. My other roles in the platform include running research collaborations and supporting other researchers in the form of consultations, mentorships, workshops, and user-friendly software development for wet-lab researchers.  My work in the last 15 years spanned various topics such as comparative genomics, protein sequence analysis in the context of molecular interactions and disease mechanisms, RNA bioinformatics, and omics data science with a focus on development of reproducible software for the bioinformatics community.

\section{Experience}
\cventry{2015--Present}{Bioinformatics Scientist}{Max-Delbrück-Center for Molecular Medicine}{Berlin, Germany}{}{}
\cventry{2014--2015}{Postdoctoral Fellow}{European Molecular Biology Laboratory}{Heidelberg, Germany}{}{}

\section{Education}
\cventry{2011--2014}{Ph.D. in Bioinformatics}{European Molecular Biology Laboratory}{Heidelberg, Germany}{}{}
\cventry{2008--2010}{M.Sc. in Bioinformatics}{Simon Fraser University, BC Cancer Agency}{Vancouver, Canada}{}{}
\cventry{2003--2008}{B.Sc. in Biological Sciences}{Sabanci University}{Istanbul, Turkey}{}{}

\section{Profiles}
\cvitemwithcomment{Google Scholar}{\href{https://scholar.google.com/citations?user=YEZr1LUAAAAJ&hl=en}{scholar.google.com/YEZr1LUAAAAJ}}{}
\cvitemwithcomment{GitHub}{\href{https://github.com/borauyar}{github.com/borauyar}}{}
\cvitemwithcomment{ORCID}{\href{https://orcid.org/0000-0002-3170-4890}{orcid.org/0000-0002-3170-4890}}{}
\cvitemwithcomment{LinkedIn}{\href{https://www.linkedin.com/in/bora-uyar-11050425/}{linkedin.com/in/bora-uyar-11050425}}{}

\section{Selected Publications}
\cvitem{Nature Communications, 2025}{\href{https://www.nature.com/articles/s41467-025-63688-5}{Flexynesis: A deep learning toolkit for bulk multi-omics data integration for precision oncology and beyond}}
\cvitem{Genome Biology, 2022}{\href{https://genomebiology.biomedcentral.com/articles/10.1186/s13059-022-02683-1}{Identifying tumor cells at the single-cell level using machine learning}}
\cvitem{Cell Reports, 2021}{\href{https://www.sciencedirect.com/science/article/pii/S2211124721003028}{Parallel genetics of regulatory sequences using scalable genome editing in vivo}}
\cvitem{Aging Research Reviews, 2020}{\href{https://www.sciencedirect.com/science/article/pii/S1568163720302919}{Single-cell analyses of aging, inflammation and senescence}}
\cvitem{GigaScience, 2018}{\href{https://academic.oup.com/gigascience/article/7/12/giy123/5114263}{PiGx: reproducible genomics analysis pipelines with GNU Guix}}
\cvitem{Cell, 2018}{\href{https://www.sciencedirect.com/science/article/pii/S0092867418310353}{Mutations in Disordered Regions Can Cause Disease by Creating Dileucine Motifs}}
\cvitem{Nucleic Acids Research, 2017}{\href{https://academic.oup.com/nar/article/45/10/e91/3038237}{RCAS: an RNA centric annotation system for transcriptome-wide regions of interest}}

\section{My research in the news}
\cvlistitem{\href{https://www.mdc-berlin.de/news/press/using-deep-learning-precision-cancer-therapy}{Using deep learning for precision cancer therapy  (mdc-berlin.de)}}
\cvlistitem{\href{https://www.mdc-berlin.de/news/news/new-textbook-computational-genomics}{New textbook for computational genomics (mdc-berlin.de)}}
\cvlistitem{\href{https://www.bionity.com/en/news/1178091/searching-the-sewers-for-viruses.html}{Searching the sewers for viruses (bionity.com)}}
\cvlistitem{\href{https://www.bionity.com/en/news/1176478/ai-identifies-cancer-cells.html}{AI identifies cancer cells (bionity.com)}}
\cvlistitem{\href{https://www.genengnews.com/news/parallel-genome-editing-in-microscopic-worms-maps-regulatory-genomic-elements-to-physiology/}{Parallel Genome Editing in Microscopic Worms Maps Regulatory Genomic Elements to Physiology (genengnews.com)}}
\cvlistitem{\href{https://www.sciencedaily.com/releases/2018/09/180906141632.htm}{Stray proteins cause genetic disorders (sciencedaily.com)}}
\cvlistitem{\href{https://www.bionity.com/en/news/1162354/a-genetic-chaperone-for-healthy-aging.html}{A genetic chaperone for healthy aging? (bionity.com)}}
\end{document}

